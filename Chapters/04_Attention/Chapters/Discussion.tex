\section{Discussion}
This study investigated the effects of spatial attention on the BOLD signal measured from individual layers in early visual cortex. Focusing first on the overall amplitude of the BOLD response in all layers combined, we found that attending to a stimulus reliably and substantially increased the BOLD signal in early visual areas, both when a stimulus was presented to the observer and in the absence of physical stimulation (cf. \cite{Kastner1999,Murray2008,Li2008}). Moreover, and much in line with earlier results on layer-specific activation patterns in visual cortex (\cite{Polimeni2010,Koopmans2010}), we observed a general increase in activation towards the superficial layers - one that is commonly explained by gradient echo being more susceptible to the draining veins on the pial surface. Interestingly, and much to our surprise, we observed no differential activity in the individual layers when comparing between top-down (attention-driven) and bottom-up (stimulus-driven) activity - a finding that stands in notable contrast to previous observations \cite{Kok2016}. We identify several potential reasons for this absence of layer specific differentiation that we outline and discuss below.
	
One possibility is that our data are simply insufficiently robust for showing a significant difference in activity across depth between the two conditions. It is well known that the BOLD signal includes multiple sources of noise related to both MRI scanner and participant, and this holds especially true for signals recorded at the sub-millimeter scale. For example, at a resolution this high, even the smallest movement of the participant may cause additional blurring of the data, with potentially detrimental effects on the signal-to-noise ratio. For this reason, we collected data from 17 participants - a sample size much larger than typical in attention-based fMRI studies at standard spatial resolution (cf. N=4-6 in\cite{Kastner1999,Kamitani2005,Jehee2011}), and it is even in the larger range for layer-based fMRI studies at high resolution (cf. N=6 in \cite{Polimeni2010}, N=4 in \cite{Muckli2015}, N=10 in \cite{Kok2016}). To minimize the effects of various sources of noise, we took great care in measuring and removing physiological artifacts, and further improved existing layer extraction techniques by developing a novel spatial general linear model that separates laminar signal from different layers instead of sampling a mixed interpolation of the layers. Additionally, we ensured that similar results were obtained using more conventional layer-extraction procedures. Indeed, the combined success of these procedures is well illustrated by the effect sizes observed in the current study for both stimulus presentation (4.5\%, 3.3\%, 2.8\% in V1, V2 and V3) and attention (0.41\%, 0.64\%, 0.59\% in V1, V2 and V3), which are comparable or higher to those reported in previous publications \cite{Murray2008,Jehee2011}. There are, however, some differences in experimental design between our study and previous laminar investigations that could potentially account for the incongruity in results. Because we were interested in the degree to which top-down processes could be dissociated from feed forward stimulation with fMRI, we directly contrasted between these two conditions in our analyses. Previous studies, on the other hand, have focused on top-down activity in isolation (e.g. \cite{Muckli2015, Kok2016}), or used multi-voxel pattern analyses - rather than overall BOLD amplitude, to compare between conditions \cite{Muckli2015}. \cite{Kok2016}, for example, directly compared between the overall activation levels in individual cortical layers, and found a significant difference in BOLD activity due to recurrent signals that were evoked by an illusory stimulus. It will be interesting for future studies to address the degree to which these experimental factors can account for the disagreement in results between our and previous work.

An alternative explanation for the incongruity in results could lie in attention itself, which may be mediated by mechanisms distinct from previously investigated processes. That is, previous work using high-resolution fMRI focused not on spatial attention, but rather on figure-ground segregation \cite{Kok2016} and other non-classical receptive field effects in cortex \cite{Muckli2015}. It is conceivable that these modulatory processes operate on the individual cortical layers in a manner dissimilar from the attentional mechanisms studied here. It is known from primate studies, for example, that attention increases the response gain of neurons in visual cortex \cite{Treue1999,MartinezTrujillo2004} - such an increase in attentional gain could lead to general enhancements in neural activity irrespective of cortical layer, as we have observed here.
	
Regardless of the potential reasons for the disparity between current and previous results, we believe our study presents an important message to a field that is currently in its nascent stages of development. We hope that the results and procedures detailed here will help move the field forward and resolve which experimental parameters are paramount, and which are not, to detecting differential activity between individual layers in human visual cortex with high-resolution fMRI.
