
The field of neuroimaging is rapidly adopting a more reproducible approach to data acquisition and analysis. Data structures and formats are being standardised and data analyses are getting more automated. However, as data analysis becomes more complicated, researchers often have to write longer analysis scripts, spanning different tools across multiple programming languages. This makes it more difficult to share or recreate code, reducing the reproducibility of the analysis. 
We present a tool, Porcupine, that \change{allows the construction of analyses in a graphical user interface, and also automatically produces analysis code.}{constructs one's analysis visually and automatically produces analysis code.} The graphical representation improves understanding of the performed analysis, while retaining the flexibility of modifying the produced code manually to custom needs. Not only does Porcupine produce the analysis code, it also creates a shareable environment for running the code in the form of a Docker image. Together, this forms a reproducible way of constructing, visualising and sharing one's analysis. Currently, Porcupine links to Nipype functionalities, which in turn accesses most standard neuroimaging analysis tools. \change[Reviewer 2]{With Porcupine, we bridge the gap between a conceptual and an implementational level of analysis and thus create reproducible and shareable science. We give the researcher a better oversight of their processing pipeline, both while developing and communicating their work. This will reduce the threshold at which less expert users can generate reusable pipelines.}{Our goal is to release researchers from the constraints of specific implementation details, thereby freeing them to think about novel and creative ways to solve a given problem. Porcupine improves the overview researchers have of their processing pipelines, and facilitates both the development and communication of their work. This will reduce the threshold at which less expert users can generate reusable pipelines. With Porcupine, we bridge the gap between a conceptual and an implementational level of analysis and make it easier for researchers to create reproducible and shareable science.} 
We provide a wide range of examples and documentation, as well as installer files for all platforms on our website: \url{https://timvanmourik.github.io/Porcupine}. Porcupine is free, open source, and released under the GNU General Public License v3.0.