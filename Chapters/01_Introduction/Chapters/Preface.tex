\section*{Preface}
`How does that work?' This may be the fundamental question of the natural sciences.
Over the centuries, scientists have discovered ever smaller particles: from molecules, to atoms, to quarks. On the other end of the spectrum, we have learned more and more about our solar system, galaxy, and the entire universe. And somewhere in between, a set of awkwardly arranged molecules forms you: a living, breathing and thinking human being.
Now, you are certainly not the only thing in the universe of academic interest, but there is something unique about phenomena found at this level: the fact that it feels like you are not merely at the whim of natural forces bouncing you around, but that you can exert control over your movements; the fact that it feels like anything at all. There must be a way that these feelings are instantiated by our molecules, by our cells, and by our brain.
To get a better grip on how humans function, the field of neuroscience has a `from molecule to man' approach. In this thesis, we will zoom in on a tiny piece of this puzzle: can we better understand communication between different regions in the brain by looking at MRI brain scans in even greater detail than is normally done?
